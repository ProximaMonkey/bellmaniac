\section{Code Generation}
\label{codegen}

We built a compiler for programs in Bellmania language that generates efficient C++ code parallelized with Intel Cilk constructs. The compiler uses type information from the development to improve the
quality of generated code. From the types of sub-terms corresponding
to array indices, the compiler extracts information about what region
of the array is read by each computation, and from the types of $\lambda$-bound index variable it constructs loops that write to
the appropriate regions. The compiler utilizes the SMT solver to infer dependency constraints as in~\cite{JACM67/Karp},
and figures out the direction of each loop (ascending or descending). 
In addition, inter-quadrant dependencies can be used to determine which computations can be run in parallel (at the granularity of
function calls) based on a fork-join model;
two calls are considered non-conflicting if each write region is disjoint from the others' read and write regions.
Type information is heavily utilized here: occurrences of array variables signify reads,
so their types denote (an over-approximation of) the region being ``read from'',
whereas return types denote the region being ``written to''.
Disjointness can be decided using propositional logic through the predicate abstraction induced by type qualifiers.

The compiler also employs more traditional optimization techniques
to further simplify the code and improve its running time:
(1) lifts conditionals to loop bounds via simple interval analysis, %to optimize the loop bounds and performs some traditional compiler %transformations to reduce the number of iterations and comparisons,
(2) eliminates redundant iterations and comparisons that can be resolved at compile time,
(3) identifies loops that read non-contiguous memory blocks and applies copy optimization~\cite{ASPLOS91/Lam} automatically to better utilize caches. Examples of generated code are included in the supplemental material with this submission,
and 
(4) follows a simple heuristic that inserts loop vectorization directives.

 
%Use Cilk for parallelization - references later
%Using SMT solver and type information to find directions of loops for fixed point computations
%Using type information and SMT solvers to parallelize different components
%Using type information to optimize the loop bounds, reduce checking of guards and do copy optimization
%Traditional compiler transformations

%code can be found in repo
%JACM67: Recurrences to loops: http://dl.acm.org/citation.cfm?id=321418
%ASPLOS1991: copy optimization: http://suif.stanford.edu/papers/lam-asplos91.pdf