\section{Related Work}
\label{related}

Classical work by Smith \etal~\cite{AI85/Smith} presents rule-based transformation, stringing it
tightly with program verification. This lay the foundation for semi-automatic programming~\cite{CPS91/Blaine,TSE90/Smith,TPHOLs96/Butler}.
More recently, a similar approach was introduced into Leon~\cite{OOPSLA13/Kneuss}, leveraging deductive
tools as a way to boost CEGIS, thereby covering more programs. Bellmania takes a dual approach, where
automated techniques based on SMT are leveraged to support and improve deductive synthesis.

Fiat~\cite{POPL15/Delaware} is another recent system that admits stepwise transformation of specifications
into programs via a refinement calculus. While Bellmania offloads proofs to automated solvers,
Fiat formalizes refinements using the Coq proof assistant. The user is then responsible of proving
the correctness of the derivation using a library of symbolic proof tactics.

Our ``$\big/$'' operator can be compared to the separating disjunction ``$\ast$'' of Separation Logic~\cite{LICS02/Reynolds},
used to frame parts of the dynamic heap (which can be thought of as one large array),
in particular while checking that a program only accesses the parts allocated to it in its precondition.
While $\ast$ has the semantics of an existentially quantified predicate, Bellmania uses type qualifiers
to explicitly specify a formula defining each part. In this sense, it is more closely related to
Region Logic~\cite{ECOOP08/Banerjee}. These formulas make encoding in first-order logic straightforward,
and the use of Liquid Types allows for any number of dimensions and for decidable checking of domain inclusion
and disjointness.
