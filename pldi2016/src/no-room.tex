\noindent
\exampleTitle \begin{comment}\subsection{Example}\end{comment}

For simplicity of the example, we assume that the input to the Simplified Arbiter problem
satisfies triangle inequalities:
%
\begin{equation}
w_{pij} \leq w_{pkj} + w_{kij} \qquad w'_{qji} \leq w'_{qki} + w'_{kji}
\label{equ:triangle}
\end{equation}
%
for all (appropriately typed) $p<k<i$, ~$q<k<j$.

\begin{comment}
\label{annex:more tactics} % make TeXlipse happy
\end{comment}

\medskip
Starting from the specification in \eqref{lang:arbiter spec}, Richard applies Synth to turn
$\fix (\theta\,i\,j\mapsto\square/\min\langle\cdots\rangle)$ into $\fix\theta\,i\,j\mapsto\min\langle\square,\cdots\rangle$.
While not absolutely necessary, we will see that it makes some expressions easier to handle later
on.

Distributivity and Associativity (refer to \Cref{annex:more tactics}) are used to obtain the tactic parameter $f_1$;
the proof is deferred until Synth is applied so that the prover can use the extra context.

\begin{tacticbox}{Synth}
  \begin{array}{@{} l @{} l @{} l @{}}
       h_1=\theta\,i\,j\mapsto{}
	      & \lspan2{0|_{i=0\land j=0} ~\big/~ w'_{0j0}|_{i=0} ~\big/~ w_{0i0}|_{j=0} ~\big/~} \\
	      & \min\,\langle~ & \min p\mapsto\theta_{pj}+w_{pij}, \\
	      & & \min q\mapsto\theta_{iq}+w'_{qji} ~\rangle \\
       f_1=\theta\,i\,j\mapsto{}
	      & \min\,\langle~ & 0|_{i=0\land j=0} ~\big/~ w'_{0j0}|_{i=0} ~\big/~ w_{0i0}|_{j=0}, \\
	      & & \min p\mapsto\theta_{pj}+w_{pij}, \\
	      & & \min q\mapsto\theta_{iq}+w'_{qji} ~\rangle
  \end{array}
\end{tacticbox}

\begin{equation}
  \renewcommand\arraystretch{1.5}
  \begin{array}{@{\!\!\!}l@{}l@{}l@{}}
    G ~=~ \fix \theta\,i\,j\mapsto{}
	      & \min\,\langle~ & 0|_{i=0\land j=0} ~\big/~ w'_{0j0}|_{i=0} ~\big/~ w_{0i0}|_{j=0}, \\
	      & & \min p\mapsto\theta_{pj}+w_{pij}, \\
	      & & \min q\mapsto\theta_{iq}+w'_{qji} ~\rangle
  \end{array}
\end{equation}

He then applies Let Insertion, followed by Stratify, to separate the base case and obtain a more general 
form, similar to {\it Compute} of \eqref{intro:breakdown}.

\begin{tacticbox}{Let}
  \begin{array}{l@{}l}
   e[\square] ~=~ \theta\,i\,j\mapsto\min\langle~ & \square, \\
      & \min p\mapsto\theta_{pj}+w_{pij}, \\
      & \min q\mapsto\theta_{iq}+w'_{qji} ~\rangle \\
   \lspan2{~~t ~=~
      0|_{i=0\land j=0} ~\big/~ w'_{0j0}|_{i=0} ~\big/~ w_{0i0}|_{j=0}}
  \end{array}
\end{tacticbox}

\begin{equation}
  \renewcommand\arraystretch{1.5}
  \begin{array}{@{}l@{}l@{}l@{}l@{}}
    G ~=~ & \fix \big(\,& \lspan2{(\theta\,i\,j\mapsto 0|_{i=0\land j=0} ~\big/~ w'_{0j0}|_{i=0} ~\big/~ w_{0i0}|_{j=0})\applt} \\
	      & & z\,\theta\,i\,j \mapsto \min\,\langle~& z_{\theta ij}, \\
	      & & & \min p\mapsto\theta_{pj}+w_{pij}, \\
	      & & & \min q\mapsto\theta_{iq}+w'_{qji} ~\rangle\,\big)
  \end{array}
\end{equation}

\begin{tacticbox}{Stratify}
  \begin{array}{l}
       f ~=~ \theta\,i\,j\mapsto 0|_{i=0\land j=0} ~\big/~ w'_{0j0}|_{i=0} ~\big/~ w_{0i0}|_{j=0} \\
       g ~=~ z\,\theta\,i\,j\mapsto \min\langle z_{\theta ij},\cdots\rangle
  \end{array}
\end{tacticbox}

\begin{equation}
  \renewcommand\arraystretch{1.5}
  \begin{array}{@{}l@{}l@{}l@{}}
    G ~=~ & \lspan2{\fix \big(\theta\,i\,j\mapsto
	              0|_{i=0\land j=0} ~\big/~ w'_{0j0}|_{i=0} ~\big/~ w_{0i0}|_{j=0}\big)\applt} \\
	      & \psi\mapsto \fix \theta\,i\,j\mapsto\min\,\langle~ & \psi_{ij} \\
	      & & \min p\mapsto\theta_{pj}+w_{pij}, \\
	      & & \min q\mapsto\theta_{iq}+w'_{qji} ~\rangle
  \end{array}
\end{equation}

Setting the base case aside, let $A^{IJ}$ be the second term,
where the superscript parameterizes the types of $i$, $j$ and $p$, $q$.

\begin{equation}
  \renewcommand\arraystretch{1.2}
  \begin{array}{@{}l@{}l@{}c@{}c@{}l@{}l@{}}
    A^{^{IJ}} ~=~ 
	      & \psi\mapsto \fix \theta\,& i & j & \mapsto\min\,\langle~ & \psi_{ij} \\
	      & & ^{^{(I)}} & ^{^{(J)}} & & \min \vtyped p I \mapsto\theta_{pj}+w_{pij}, \\
	      & & & & & \min \vtyped q J \mapsto\theta_{iq}+w'_{qji} ~\rangle
  \end{array}
  \label{tactics:arbiter phase A}
\end{equation}

Vertical typeset was used to save some horizontal space, but $\vtyped v\T$
should be read as just $v:\T$.

\medskip
This is exactly what Richard was after. $A^{IJ}$ constitutes the first phase
of the divide-and-conquer formalization.

\medskip
\hrule
\bigskip
