\section{More Tactics}
\label{annex:more tactics}

\tacticdef{Shrink} \label{tactics:Shrink}
\[f ~=~ f :: \T\] 

Used to specify tighter qualifiers for the type of a sub-term.

\Obligations tactic.

For arrow-typed terms, this essentially requires to prove that $f$ 
is only defined for arguments in the domain of $\T$, and that the values are in the range of $\T$.
This can be seen as a special case of {\sf Slice} with $r=1$, with the additional
feature of specifying the range as well.

\tacticdef{Associativity}
\[\reduce\big\langle \reduce\langle\overline x_1\rangle,\cdots, \reduce\langle\overline x_r\rangle\big\rangle ~=~ \reduce \langle \overline x_1, \cdots, \overline x_r\rangle\]
%
where $\reduce$ is a built-in aggregation ($\min$, $\max$, $\Sigma$), 
and $\overline x_i$ are lists of terms (of the same type).
If any of $\overline x_i$ is of length one, $\reduce\langle\overline x_i\rangle$ can be
replaced by $\overline x_i$.

\Obligations none.

\tacticdef{Distributivity}\vspace{3mm}
Let $e$ be an expression with a hole, $e[\square] = (\cdots \square \cdots)$.
%
\[\renewcommand\arraystretch{1.2}
  \begin{array}{@{}r@{~}c@{~}l@{}}
    e[t_1/\cdots/t_r] &=& e[t_1] / \cdots / e[t_r] \\
    e[t_1/\cdots/t_r] &=& \reduce\langle e[t_1],\cdots,e[t_r]\rangle \\
    \reduce e[t_1/\cdots/t_r] &=& \reduce\langle \reduce e[t_1],\cdots,\reduce e[t_r]\rangle
  \end{array}\]

This tactic provides several alternatives for different uses of aggregations.
Clearly, $\big/$ does not distribute over any expression; we give just a few examples
where this tactic is applicable.
\begin{itemize}
  \item $(x/y)+1 ~=~ (x+1)~/~(y+1)$
  \item $x/0 ~=~ \max\langle x,0\rangle$ ~(for $x:\N$)
  \item $\min \big([f]_{J_0}~\big/~[f]_{J_1}\big) ~=~
         \min\left\langle \min [f]_{J_0} ~,~ \min [f]_{J_1}\right\rangle$
\end{itemize}

\Obligations tactic.

\tacticdef{Elimination}
\[e[t] ~=~ e[\bot]\]
%
Used to eliminate a sub-term that is either always undefined or has no effect
in the context in which it occurs.

\Obligations tactic.

\tacticdef{Let Insertion}\vspace{3mm}
Let $e$ be an expression with a hole, $e[\square] = (\cdots x_1 \mapsto \cdots x_k\mapsto \cdots \square \cdots)$, 
where $x_{1..k}\mapsto$ are abstraction terms enclosing $\square$. The bodies may contain arbitrary terms
in addition to these abstractions.
%
\[\renewcommand\arraystretch{1.2}
  \begin{array}{@{}r@{}c@{}l@{}}
  e[t] & ~=~ & (\overline{x}\mapsto t) ~\applt~ z\mapsto e[z\,\overline{x}] \\
  e[\reduce \langle\overline{a},\overline{b}\rangle] & ~=~ &
     (\overline{x}\mapsto \reduce\langle\overline{a}\rangle) ~\applt~ z\mapsto e[\reduce\langle z\,\overline{x}, \overline{b}\rangle]
  \end{array}\]
%
where $\overline{x}=x_{1..k}$, and $z$ is a fresh variable.
This tactic also has a special version that involves $\reduce$.
The items in $\langle \overline{a},\overline{b}\rangle$ may be interleaved, since
$\min,\max,\Sigma$ all happen to be commutative.\footnote{If non-commutative functions get added in the future, then this will change into $\langle \overline{a}, \overline{b}, \overline{c}\rangle$ non-interleaving, with the right hand side being $(\overline{x}\mapsto \reduce\langle\overline{b}\rangle) ~\applt~ z\mapsto e[\reduce\langle \overline{a}, z\,\overline{x}, \overline{c}\rangle]$.}

\Obligations tactic, if $z$ occurs free in $e$; otherwise none.

\tacticdef{Let Insertion [$\reduce$]}
%
\[e[\reduce \langle\overline{a},\overline{b}\rangle] ~=~ 
  (\overline{x}\mapsto \reduce\langle\overline{a}\rangle) ~\applt~ z\mapsto e[\reduce\langle z\,\overline{x}, \overline{b}\rangle]\]
%
where $\overline{x}=x_{1..k}$, and $z$ a fresh variable.

\Obligations tactic, if $z$ occurs free in $e$; otherwise none.

\tacticdef{Padding}
\[t ~=~ \big(t~/~f_1/\cdots/f_r\big) :: \T\]
%
where $\T$ is the type of $t$. This tactic is commonly used with Let insertion,
to make the type of a sub-computation match the type of the entire term.

\Obligations tactic.

\tacticdef{Pull Out}\vspace{3mm}
For $e[\square]$ as defined previously:
\[z ~=~ \overline{x}\mapsto t\]
%
where $z$ is a fresh variable.

Similar to Let Insertion, but does not change the original term; instead,
it is used to single out and name a particular expression $t$, preserving the context
in which it occurs in $e[t]$. It is not a tactic {\it per se}, as it does not
actually effect any transformation on $e[t]$; instead, it is designed to increase
readability of the development and simplify successive human-computer interaction.

\section{More Proofs}

We prove the corollary from \Cref{tactics:soundness}.

\medskip\noindent
{\bf Corollary.~} Let $f : \T\to\T$; if either $f :: \T\to\Y = f :: \Y\to\Y$ or $f :: \Y\to\T = f :: \Y\to\Y$, 
then $(\fix f)::\Y = \fix (f :: \Y\to\Y)$.

\begin{proof}

 For the first case, assume $\theta=\fix f$,

\[\renewcommand\arraystretch{1.3}
  \begin{array}{l}
   \theta::\Y = (\theta\applt f)::\Y = \theta\applt(f::\T\to\Y) = \\
   \qquad=\theta\applt(f::\Y\to\Y) = (\theta::\Y)\applt(f::\Y\to\Y)
 \end{array}\]

This means that $\theta::\Y = \fix(f::\Y\to\Y)$, as desired.
For the second case, from domain theory we know that $\fix f = f^k\bot$ for some $k\geq 1$.
We prove by induction that $f^k\bot = (f :: \Y\to\Y)^k\bot$.

For $k=1$, \[f\,\bot = f\,(\bot :: \Y) = (f :: \Y\to\T)\,\bot = (f :: \Y\to\Y)\,\bot\]

Assume $f^k\bot = (f :: \Y\to\Y)^k\bot$, then definitely $f^k\bot = f^k\bot :: \Y$.
Therefore, 
\[\renewcommand\arraystretch{1.3}
  \begin{array}{l}
    f^{k+1}\bot = (f^k\bot)\applt f = (f^k\bot::\Y)\applt f = \\
    \qquad = (f^k\bot)\applt(f::\Y\to\T) = \\
    \qquad = ((f :: \Y\to\Y)^k\bot) \applt (f::\Y\to\Y) = \\
    \qquad = (f::\Y\to\Y)^{k+1}\bot
  \end{array}
\]

From this we learn that $\fix f = \fix(f :: \Y\to\Y) = (\fix f) :: \Y$.
\end{proof}
