\section{Conclusion}
\label{conc}

The examples in this paper show that a few well-placed tactics can cover a wide range
of program transformations. The introduction of solver-aided tactics allowed us to make
the library of tactics smaller, by enabling the design of higher-level, more generic
tactics. Their small number gives the hope that end-users with some mathematical background
will be able to use the system without the steep learning curve that is usually associated
with proof assistants. This can be a valuable tool for algorithms research.

\begin{comment}
Moreover, limiting the number of tactics shrinks the space in which to search for programs,
so that an additional level automation may be achieved via AI or ML methods. As more
developments are done by humans and collected in a database, those algorithms would become
more adept in predicting the next step of the construction.
\end{comment}

But solver-aided tactics should not be seen as specific to divide-and-conquer algorithms,
or even to algorithms. The same approach can be applied to other domains. Domain knowledge
can be used to craft specialized tactics, providing users with the power to use a high-level
DSL to specify their requirements, without sacrificing performance.
