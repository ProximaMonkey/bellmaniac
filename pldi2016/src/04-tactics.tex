\section{Tactics}
\label{tactics}

We now define the method by which that our framework transforms program terms, by means of \newterm{tactics}.
A tactic is a scheme of equalities that can be used for rewriting.
When applied to a program term, any occurrence of the {\bf left-hand side} is replaced by the {\bf right-hand side}.%
\footnote{This is also a standard convention in Coq, for example.}
A valid application of a tactic is an instance of the scheme that is well-typed and logically valid
(that is, the two sides have the same interpretation in any structure that interprets the free
variables occurring in the equality).

The application of tactics yields a sequence of program terms, each of which is checked to
be equivalent to the previous one. We refer to this sequence by the name \newterm{development}.

We associate with each tactic some \newterm{proof obligations}, listed after the word \textbf{\textit{Obligations}}
in the following paragraph.
When applying a tactic instance, these obligations are also instantiated and given to an automated prover. 
If verified successfully, they entail the validity of the instance. 
Clearly the tactic itself can be used as its proof obligation, if it is easy enough to prove automatically; 
in such cases we write ``\textbf{\textit{Obligations}:} tactic.''

The following are the major tactics provided by our framework. 
More tactic definitions are given in the appendix.

\newcommand\Obligations{\medskip\noindent\textbf{\textit{Obligations}:} }
\newcommand\reduce{\operatorname{reduce}}
\newcommand\listConcat{{\scriptstyle \,++\,}}

\theoremstyle{definition}
\newtheorem{tactic}{Tactic}

\newcommand\tacticdef[1]{\subsection*{\sf\larger #1}\vspace{-3mm}}
\newcommand\tacticdefcompact[1]{\medskip\noindent{\sf\larger #1}\medskip\hfill}

\tacticdefcompact{Slice} \label{tactics:Slice}
$f ~=~ \big[f\big]_{X_1} ~\Big/~ \big[f\big]_{X_2} ~\Big/ ~\cdots~ \Big/~ \big[f\big]_{X_r}$\hspace{10mm}

This tactic partitions a mapping into sub-regions. Each $X_i$ may be a cross product ($\times$)
according to the arity of $f$.

\Obligations tactic.

Informally, the recombination expression is equal to $f$
when $X_{1..r}$ ``cover'' all the defined points of $f$ (also known as the \newterm{support} of $f$).

\medskip
\tacticdefcompact{Stratify}
$\fix (f\applt g) ~=~ (\fix f) ~\applt~ \big(\psi\mapsto \fix (\widehat\psi\applt g)\big)$
\\
where $\widehat\psi$ abbreviates $\theta\mapsto\psi$, with fresh variable $\theta$.

This tactic is used to break a long (recursive) computation into simpler sub-computations.
$\psi$ may be fresh, or it may reuse a variable already occurring in $g$, rebinding those occurrences.
An example is required to understand why this is useful.

Let $h=\theta\mapsto\big([t_0]_{I_0}\big/[t_1]_{I_1}\big)$, where $t_{0,1}$ are terms with free variable $\theta$.
Stratification is done by setting $f=\theta\mapsto t_0$ and
 $g=f\,\theta\mapsto [f\,\theta]_{I_0}\big/[t_1]_{I_1}$. We get $\fix h=\fix(f\applt g)$; {\sf~Stratify}
breaks it into $\fix f$ and $\psi\mapsto \fix (\widehat\psi\applt g)$,
which get simplified with $\beta$-reduction, giving the expression the form:

\vspace{-5mm}
\[\big(\fix(\theta\mapsto t_0)\big) ~\applt~ 
  \Big(\psi\mapsto \fix \big(\theta\mapsto[\psi]_{I_0}\big/[t_1]_{I_1}\big)\Big)\]

\vspace{-2mm}
This precisely encodes our intuition of computing the fixed point of $t_0$ first,
then $t_1$ based on the result of $t_0$.

\Obligations Let $h=f\applt g$ and $g'=\psi\mapsto\widehat\psi\applt g$. Let $\theta,\zeta$ be
fresh variables.
\begin{equation}
\renewcommand\arraystretch{1.5}
\begin{array}{l@{\qquad}l}
f\,(g'\,\zeta\,\theta) ~=~ f\,\zeta &
g'\,(f\,\theta)\,\theta ~=~ h\,\theta
\end{array}
\label{tactics:Stratify obligations}
\end{equation}

Proof is given in \Cref{tactics:soundness}.

\tacticdefcompact{Synth} \label{tactics:Synth}
$\fix\big(h_1 ~\big/~ \cdots ~\big/~ h_r\big) ~=~ 
  f_1 :: \T_1 ~\big/~ \cdots ~\big/~ f_r :: \T_r$

This tactic is used to generate recursive calls to sub-programs. For $i=1..r$, $f_i$
is one of the following: $\fix h_i$, $h_i\,\psi$, or $t\,\psi$, where $\psi$ is some
variable and $t$ is a term corresponding to a previously defined subroutine
($A$, $B$, $C$ in the example).
Bellmania chooses these values automatically (see \Cref{tactics:synthesis}),
but the user may override it.

\newcommand\Y{\mathcal{Y}}

\Obligations Let $h=h_1/\cdots/h_r$, and let $\T\to\T$ be the shape of $h$. 
  For each $f_i$, depending on the form of $f_i$:
\begin{itemize}
  \item If $f_i \cong \fix f$ --- \\
    \rule{0pt}{12pt}
    $h :: (\T \to \Y) = h :: (\Y \to \Y) = g :: (\Y \to \T)$
    for some $\Y$ which is a subtype of $\T$ and a supertype of $\T_i$.
  \item If $f_i$ does not contain any ``$\fix$'' terms ---\\
    \rule{0pt}{12pt}
    $h\,(h\,\theta) :: \T_i = f_i :: \T_i$ for a fresh variable $\theta$.
\end{itemize}

We use $\cong$ to denote syntactic congruence up to $\beta$-reduction.

\begin{comment}
\Obligations Let $h=h_1/\cdots/h_r$, let $\overline\theta\!=\!\theta_{1..r}$ be $r$ fresh variables, and let
$f = \theta_{1..r} \mapsto (f_1\,\theta_1)::\T_1/\cdots/(f_r\,\theta_r)::\T_r$.
\begin{itemize}
  \item $\T_{1..r}$ are disjoint mappings.
  \item {\bf Either}\quad $h\,(f\,\overline\theta) = f\,\overline\theta$ \\{\bf or}\qquad
  $\begin{array}[t]{l} h\,\theta=\theta ~\limplies~ (f_i\,\theta :: \T_i)=\theta :: \T_i ~, \\
  \theta::\T_1~/~\cdots~/~\theta::\T_r=\theta\end{array}$
\end{itemize}

(We give two alternatives, as the first is usually easier to prove, but may hold in less cases)
\end{comment}


\newenvironment{tacticbox}[1]{\begin{center}
  \begin{tabular}{|@{~~~~}l@{~~~~}|}\hline
    \rule{0pt}{2.3ex}\underline{\sf \,#1\,}\\[.4em]$}
  {$\\[-1em] \\[.3ex] \hline \end{tabular} \end{center}}


\subsection{Synthesis-supported {\sf Synth} Tactic}
\label{tactics:synthesis}

As mentioned in \Cref{intro,overview}, the user is assisted by automatic
inference while applying tactics. In particular, the {\sf Synth} tactic requires
the user to specify a subroutine to call and parameters to call it with.
In addition, the subtype $\Y$ is required to complete the correctness proof.
To automate this task, Bellmania employs {\cegis}, a software synthesis technique
implemented in the tool {\Sketch}~\cite{STTT13/Solar-Lezama}. The proof obligations, along with the possible
space of parameter assignments taken from the set of sub-types defined during
{\sf Slice}, are translated to {\Sketch}. Since {\Sketch} uses bounded domains,
the result is then verified using full SMT.

While the size of the search space is not huge, typically in the order of \textasciitilde50
alternatives, it is usually hard for the user to figure out which exact call
should be made. Since {\sf Synth} is used extensively throughout the develoment,
This kind of automation greatly improves overall usability.

\subsection{Soundness}
\label{tactics:soundness}

\renewenvironment{proof}{\noindent{\bf Proof.~}}{}

\begin{theorem}
Let $s=s'$ be an instance of one of the tactics introduced in this section.
let $a_i=b_i$, $i=1..k$, be the proof obligations. If $\semp{a_i}=\semp{b_i}$
for all interpretations of the free variables of $a_i$ and $b_i$, then
$\semp{s}=\semp{s'}$ for all interpretations of the free variables of $s$ and $s'$.
\end{theorem}

\begin{proof}
For the tactics with \textbf{\textit{Obligations}:} tactic, the theorem is trivial.

\medskip
\noindent
{\tt >} For {\sf Stratify}, let $f$, $g$ be partial functions such that
\vspace{-.5em}
\[\renewcommand\arraystretch{1.3}
  \forall \theta,\zeta.\quad \begin{array}{l}f\,(g\,\zeta\,\theta) ~=~ f\,\zeta \quad\land\quad
  g\,(f\,\theta)\,\theta ~=~ h\,\theta
  \end{array}\quad\]
  
Assume that $\zeta = \fix f$ and $\theta = \fix (g\,\zeta)$. 
That is, $f\,\zeta = \zeta$ and $g\,\zeta\,\theta = \theta$.
Then ---
\vspace{-.5em}
\[\renewcommand\arraystretch{1.3}
  \begin{array}{l@{}l}
   h\,\theta & {}= g\,(f\,\theta)\,\theta = g\,(f\,(g\,\zeta\,\theta))\,\theta =
              g\,(f\,\zeta)\,\theta = \theta
  \end{array}\]
  
\noindent
So $\theta = \fix h$. We get $\fix h = \fix \big(g \,(\fix f)\big)$; equivalently,
\[\fix h = (\fix f) \applt \big(\psi\mapsto\fix (g\,\psi)\big)\]

Now instantiate $h$, $f$, and $g$, with $f\applt g$, $f$, and $g'$ from \Cref{tactics:Stratify obligations},
and we obtain the equality in the tactic.

\medskip
\noindent
{\tt >} For {\sf Synth}, ({\it i}) assume $f_i=\fix g$ and
\[h :: \T\to\Y = h :: \Y\to\Y = g :: \Y\to\T\]

Intuitively, $\Y$ ``cuts out'' a region of an array $\theta :: \T$ given
as input to $h$ and $g$. This area is self-contained, in the sense that
only elements in $\Y$ are needed to compute elements in $\Y$, as indicated
by the refined type $\Y\to\Y$.

Notice that from the premise follows $g :: \Y\to\T = g :: \Y\to\Y$. We use the following corollary:

\medskip\noindent
{\bf Corollary.~} Let $f : \T\to\T$; if either $f :: \T\to\Y = f :: \Y\to\Y$ or $f :: \Y\to\T = f :: \Y\to\Y$, 
then $(\fix f)::\Y = \fix (f :: \Y\to\Y)$.

Proof is included in the appendix.

\medskip
From the corollary, and for the given $h$ and $g$, we learn that $(\fix h)::\Y = \fix(h::\Y\to\Y)$,
and also $(\fix g)::\Y=\fix(g::\Y\to\Y)$. Since $h::\Y\to\Y=g::\Y\to\Y$,
we get $(\fix h)::\Y = (\fix g)::\Y$; now, $\Y$ is a supertype of $\T_i$, so $(\theta::\Y)::\T_i=\theta::\T_i$:
\[\renewcommand\arraystretch{1.3}
  \begin{array}{l}(\fix h)::\T_i = ((\fix h)::\Y)::\T_i = ((\fix g)::\Y)::\T_i = \\
    \qquad = (\fix g)::\T_i=f_i::\T_i
  \end{array}\]

{\it (ii)} Assume $h\,(h\,\theta) :: \T_i = f_i :: \T_i$ holds for any $\theta:\T$,
then in particlar, for $\theta=\fix\,h$, we get $h\,(h\,\fix h) :: \T_i = f_i :: \T_i$.
Since $h\,(h\,\fix h) = \fix h$, we obtain the conjecture $(\fix h) :: \T_i = f_i :: \T_i$.
\qed
\end{proof}

\medskip
Our reliance on the termination of $\fix$ expressions may seem conspicuous, since some of these
expressions are generated automatically by the system. However, a closer look reveals that whenever
such a computation is introduced, the set of the recursive calls it makes is a subset of those made by the existing one.
Therefore, if the original recurrence terminates, so does the new one. In any case, all the recurrences
in our development have a trivial termination argument (the indexes $i$,$j$ change monotonically between calls),
so practically, this should never become a problem.

\newcommand\vtyped[2]{\underset{\scriptscriptstyle ( #2 )}{ #1 }}

