\section{Tactics}

A \newterm{tactic} is a scheme of equalities that can be used for rewriting.
When applied to a program term, any occurrence of the left-hand side is replaced by the right-hand side.
A valid application of a tactic is an instance of the scheme that is well-typed and logically valid
(that is, the two sides have the same interpretation in any structure that interprets the free
variables occurring in the equality).

The application of tactics yields a sequence of program terms, each of which is checked to
be equivalent to the previous one. We refer to this sequence by the name \newterm{development}.

We associate with each tactic some \newterm{proof obligations}, listed after the word {\bf Obligations}
in each of the following subsections.
When applying a tactic instance,
these obligations are also instantiated and given to an automated prover. If verified successfully,
they entail the validity of the instance. Clearly the tactic itself can be used as its proof obligation,
if it is easy enough to prove automatically; in such cases we write ``{\bf Obligations:} tactic.''

\newcommand\Obligations{\medskip\noindent{\bf Obligations:} }

\subsection{Main Tactics}

\subsubsection{Slice}
\[f ~=~ f\big|_{X_1} ~\Big/~ f\big|_{X_2} ~\Big/ ~\cdots~ \Big/~ f\big|_{X_r}\]

This tactic partitions a mapping into sub-regions. Each $X_i$ may be a $\times$-expression
according to the arity of $f$.

\Obligations tactic.

Informally, the recombination expression is equal to $f$
when $X_{1..r}$ ``cover'' all the defined points of $f$ (also known as the \newterm{support} of $f$).

\subsubsection{Shrink}
\[f ~=~ f :: \T\]

Used to extra-specify the type of a sub-term.

\Obligations tactic.

For arrow-typed terms, this essentially requires to prove that $f$ is undefined anyway for
argument values outside the domain of $\T$, and that the defined values are in the range of $\T$.

\subsubsection{Stratify}
\[\fix (f\applt g) ~=~ \fix f ~\applt~ \psi\mapsto \fix (\dot\psi\applt g)\]
%
where $\dot\psi$ abbreviates $\theta\mapsto\psi$, with fresh variable $\theta$.

$\psi$ may be fresh, or it may reuse a variable already occurring in $g$, rebinding those occurrences.
The example of this section will illustrate why this is useful.

\Obligations Let $h=f\applt g$ and $g'=\psi\mapsto\dot\psi\applt g$. Let $\theta,\zeta$ be
fresh variables.
\begin{equation}
\renewcommand\arraystretch{1.5}
\begin{array}{l}
f\,(g'\,\zeta\,\theta) ~=~ f\,\zeta \\
g'\,(f\,\theta)\,\theta ~=~ h\,\theta
\end{array}
\label{tactics:Stratify obligations}
\end{equation}

Although the proof is not hard, we defer it to a later theorem.

\subsubsection{Synth}
\[\fix\Big(h_1 ~\big/~ \cdots ~\big/~ h_r\Big) ~=~ 
  (\fix f_1) :: \T_1 ~\big/~ \cdots ~\big/~ (\fix f_r) :: \T_r\]

This tactic is used to generate recursive calls to sub-programs. For $i=1..r$, $f_i$
is typically either $h_i$ or a sub-term occurring earlier in the development.

\Obligations Let $h=h_1/\cdots/h_r$, let $\overline\theta\!=\!\theta_{1..r}$ be $r$ fresh variables, and let
$f = \theta_{1..r} \mapsto (f_1\,\theta_1)::\T_1/\cdots/(f_r\,\theta_r)::\T_r$.
\begin{itemize}
  \item $\T_{1..r}$ are disjoint mappings.
  \item $h\,(f\,\overline\theta) = f\,\overline\theta$.
\end{itemize}

\subsection{Minor Tactics}

\subsubsection{Associativity}
\[\mathrm{reduce}\,(x {\scriptstyle\,++\,} y) ~=~ \mathrm{reduce}\,\langle \mathrm{reduce}\,x, \mathrm{reduce}\,y\rangle\]

\subsubsection{Distributivity}
Let $e$ be an expression with a hole, $e[\square] = (\cdots \square \cdots)$.
%
\[e[f_1/\cdots/f_r] ~=~ e[f_1] / \cdots / e[f_r]\]

\subsubsection{Elimination}
\[e[t] ~=~ e[\bot]\]

\subsubsection{Let Insertion}

Let $e$ be an expression with a hole, $e[\square] = (\cdots x_1 \mapsto \cdots x_k\mapsto \cdots \square \cdots)$, 
where $x_{1..k}\mapsto$ are abstraction terms enclosing $\square$. The bodies may contain arbitrary terms
in addition to these abstractions. Let $t$ be some other term.
%
\[e[t] ~=~ (\overline{x}\mapsto t) ~\applt~ z\mapsto e[z\,\overline{x}]\]
%
where $\overline{x}=x_{1..k}$, and $z$ is a fresh variable.


\subsection{Example \hrulefill}
\hrule
\bigskip
