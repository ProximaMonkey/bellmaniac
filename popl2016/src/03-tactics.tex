\section{Tactics}
\label{tactics}

We now define the method by which that our framework transforms program terms, by means of \newterm{tactics}.
A tactic is a scheme of equalities that can be used for rewriting.
When applied to a program term, any occurrence of the left-hand side is replaced by the right-hand side.
A valid application of a tactic is an instance of the scheme that is well-typed and logically valid
(that is, the two sides have the same interpretation in any structure that interprets the free
variables occurring in the equality).

The application of tactics yields a sequence of program terms, each of which is checked to
be equivalent to the previous one. We refer to this sequence by the name \newterm{development}.

We associate with each tactic some \newterm{proof obligations}, listed after the word \textbf{\textit{Obligations}}
in the following paragraph.
When applying a tactic instance, these obligations are also instantiated and given to an automated prover. 
If verified successfully, they entail the validity of the instance. 
Clearly the tactic itself can be used as its proof obligation, if it is easy enough to prove automatically; 
in such cases we write ``\textbf{\textit{Obligations}:} tactic.''

The following are the major tactics provided by our framework. 
More tactic definitions are given in the appendix.

\newcommand\Obligations{\medskip\noindent\textbf{\textit{Obligations}:} }
\newcommand\reduce{\operatorname{reduce}}
\newcommand\listConcat{{\scriptstyle \,++\,}}

\theoremstyle{definition}
\newtheorem{tactic}{Tactic}

\newcommand\tacticdef[1]{\subsection*{#1}}

\tacticdef{Slice} \label{tactics:Slice}
\[f ~=~ f\big|_{X_1} ~\Big/~ f\big|_{X_2} ~\Big/ ~\cdots~ \Big/~ f\big|_{X_r}\] 

This tactic partitions a mapping into sub-regions. Each $X_i$ may be a $\times$-expression
according to the arity of $f$.

\Obligations tactic.

Informally, the recombination expression is equal to $f$
when $X_{1..r}$ ``cover'' all the defined points of $f$ (also known as the \newterm{support} of $f$).

\tacticdef{Shrink} \label{tactics:Shrink}
\[f ~=~ f :: \T\] 

Used to extra-specify the type of a sub-term.

\Obligations tactic.

For arrow-typed terms, this essentially requires to prove that $f$ is undefined anyway for
argument values outside the domain of $\T$, and that the defined values are in the range of $\T$.

\tacticdef{Stratify} \label{tactics:Stratify}
\[\fix (f\applt g) ~=~ \fix f ~\applt~ \psi\mapsto \fix (\dot\psi\applt g)\]
%
where $\dot\psi$ abbreviates $\theta\mapsto\psi$, with fresh variable $\theta$.

This tactic is used to break a long (recursive) computation into simpler sub-computations.
$\psi$ may be fresh, or it may reuse a variable already occurring in $g$, rebinding those occurrences.
The example of this section will illustrate why this is useful.

\Obligations Let $h=f\applt g$ and $g'=\psi\mapsto\dot\psi\applt g$. Let $\theta,\zeta$ be
fresh variables.
\begin{equation}
\renewcommand\arraystretch{1.5}
\begin{array}{l}
f\,(g'\,\zeta\,\theta) ~=~ f\,\zeta \\
g'\,(f\,\theta)\,\theta ~=~ h\,\theta
\end{array}
\label{tactics:Stratify obligations}
\end{equation}

Although the proof is not hard, we defer it to a later theorem.

\tacticdef{Synth} \label{tactics:Synth}
\[\fix\big(h_1 ~\big/~ \cdots ~\big/~ h_r\big) ~=~ 
  (\fix f_1) :: \T_1 ~\big/~ \cdots ~\big/~ (\fix f_r) :: \T_r\]

This tactic is used to generate recursive calls to sub-programs. For $i=1..r$, $f_i$
is typically either $h_i$ or a sub-term occurring earlier in the development.

\Obligations Let $h=h_1/\cdots/h_r$, let $\overline\theta\!=\!\theta_{1..r}$ be $r$ fresh variables, and let
$f = \theta_{1..r} \mapsto (f_1\,\theta_1)::\T_1/\cdots/(f_r\,\theta_r)::\T_r$.
\begin{itemize}
  \item $\T_{1..r}$ are disjoint mappings.
  \item {\bf Either}\quad $h\,(f\,\overline\theta) = f\,\overline\theta$ \\{\bf or}\qquad
  $\begin{array}[t]{l} h\,\theta=\theta ~\limplies~ (f_i\,\theta :: \T_i)=\theta :: \T_i ~, \\
  \theta::\T_1~/~\cdots~/~\theta::\T_r=\theta\end{array}$
\end{itemize}

(We give two alternatives, as the first is usually easier to prove, but may hold in less cases)

\newenvironment{tacticbox}[1]{\begin{center}\begin{tabular}{|@{~~~~}l@{~~~~}|}\hline\rule{0pt}{2.3ex}\underline{#1}\\[.4em]$}{$\\[-1em] \\[.3ex] \hline\end{tabular}\end{center}}

\noindent
\begin{minipage}{\columnwidth}
\exampleTitle \begin{comment}\subsection{Example}\end{comment}

For simplicity of the example, we assume that the input to the Simplified Arbiter problem
satisfies triangle inequalities:
\end{minipage}
%
\begin{equation}
w_{pij} \leq w_{pkj} + w_{kij} \qquad w'_{qji} \leq w'_{qki} + w'_{kji}
\label{equ:triangle}
\end{equation}
%
for all (appropriately typed) $p<k<i$, ~$q<k<j$.

\begin{comment}
\label{annex:more tactics} % make TeXlipse happy
\end{comment}

\medskip
Starting from the specification in \eqref{lang:arbiter spec}, Richard applies Synth to turn
$\fix (\theta\,i\,j\mapsto\square/\min\langle\cdots\rangle)$ into $\fix\theta\,i\,j\mapsto\min\langle\square,\cdots\rangle$.
While not absolutely necessary, we will see that it makes some expressions easier to handle later
on.

Distributivity and Associativity (refer to \Cref{annex:more tactics}) are used to obtain the tactic parameter $f_1$;
the proof is deferred until Synth is applied so that the prover can use the extra context.

\begin{tacticbox}{Synth}
  \begin{array}{@{} l @{} l @{} l @{}}
       h_1=\theta\,i\,j\mapsto{}
	      & \lspan2{0|_{i=0\land j=0} ~\big/~ w'_{0j0}|_{i=0} ~\big/~ w_{0i0}|_{j=0} ~\big/~} \\
	      & \min\,\langle~ & \min p\mapsto\theta_{pj}+w_{pij}, \\
	      & & \min q\mapsto\theta_{iq}+w'_{qji} ~\rangle \\
       f_1=\theta\,i\,j\mapsto{}
	      & \min\,\langle~ & 0|_{i=0\land j=0} ~\big/~ w'_{0j0}|_{i=0} ~\big/~ w_{0i0}|_{j=0}, \\
	      & & \min p\mapsto\theta_{pj}+w_{pij}, \\
	      & & \min q\mapsto\theta_{iq}+w'_{qji} ~\rangle
  \end{array}
\end{tacticbox}

\begin{equation}
  \renewcommand\arraystretch{1.5}
  \begin{array}{@{\!\!\!}l@{}l@{}l@{}}
    G ~=~ \fix \theta\,i\,j\mapsto{}
	      & \min\,\langle~ & 0|_{i=0\land j=0} ~\big/~ w'_{0j0}|_{i=0} ~\big/~ w_{0i0}|_{j=0}, \\
	      & & \min p\mapsto\theta_{pj}+w_{pij}, \\
	      & & \min q\mapsto\theta_{iq}+w'_{qji} ~\rangle
  \end{array}
\end{equation}

He then applies Let Insertion, followed by Stratify, to separate the base case and obtain a more general 
form, similar to $\Ggen$ of \eqref{intro:Ggen}.

\begin{tacticbox}{Let}
  \begin{array}{l@{}l}
   e[\square] ~=~ \theta\,i\,j\mapsto\min\langle~ & \square, \\
      & \min p\mapsto\theta_{pj}+w_{pij}, \\
      & \min q\mapsto\theta_{iq}+w'_{qji} ~\rangle \\
   \lspan2{~~t ~=~
      0|_{i=0\land j=0} ~\big/~ w'_{0j0}|_{i=0} ~\big/~ w_{0i0}|_{j=0}}
  \end{array}
\end{tacticbox}

\begin{equation}
  \renewcommand\arraystretch{1.5}
  \begin{array}{@{}l@{}l@{}l@{}l@{}}
    G ~=~ & \fix \big(\,& \lspan2{(\theta\,i\,j\mapsto 0|_{i=0\land j=0} ~\big/~ w'_{0j0}|_{i=0} ~\big/~ w_{0i0}|_{j=0})\applt} \\
	      & & z\,\theta\,i\,j \mapsto \min\,\langle~& z_{\theta ij}, \\
	      & & & \min p\mapsto\theta_{pj}+w_{pij}, \\
	      & & & \min q\mapsto\theta_{iq}+w'_{qji} ~\rangle\,\big)
  \end{array}
\end{equation}

\begin{tacticbox}{Stratify}
  \begin{array}{l}
       f ~=~ \theta\,i\,j\mapsto 0|_{i=0\land j=0} ~\big/~ w'_{0j0}|_{i=0} ~\big/~ w_{0i0}|_{j=0} \\
       g ~=~ z\,\theta\,i\,j\mapsto \min\langle z_{\theta ij},\cdots\rangle
  \end{array}
\end{tacticbox}

\begin{equation}
  \renewcommand\arraystretch{1.5}
  \begin{array}{@{}l@{}l@{}l@{}}
    G ~=~ & \lspan2{\fix \big(\theta\,i\,j\mapsto
	              0|_{i=0\land j=0} ~\big/~ w'_{0j0}|_{i=0} ~\big/~ w_{0i0}|_{j=0}\big)\applt} \\
	      & \psi\mapsto \fix \theta\,i\,j\mapsto\min\,\langle~ & \psi_{ij} \\
	      & & \min p\mapsto\theta_{pj}+w_{pij}, \\
	      & & \min q\mapsto\theta_{iq}+w'_{qji} ~\rangle
  \end{array}
\end{equation}

Setting the base case aside, let $A^{IJ}$ be the second term,
where the superscript parameterizes the types of $i$, $j$ and $p$, $q$.

\newcommand\vtyped[2]{\underset{\scriptscriptstyle ( #2 )}{ #1 }}

\begin{equation}
  \renewcommand\arraystretch{1.2}
  \begin{array}{@{}l@{}l@{}c@{}c@{}l@{}l@{}}
    A^{^{IJ}} ~=~ 
	      & \psi\mapsto \fix \theta\,& i & j & \mapsto\min\,\langle~ & \psi_{ij} \\
	      & & ^{^{(I)}} & ^{^{(J)}} & & \min \vtyped p I \mapsto\theta_{pj}+w_{pij}, \\
	      & & & & & \min \vtyped q J \mapsto\theta_{iq}+w'_{qji} ~\rangle
  \end{array}
  \label{tactics:arbiter phase A}
\end{equation}

Which is exactly what Richard was after.
Vertical typeset was used to save some horizontal space, but $\vtyped v\T$
should be read as just $v:\T$.

\medskip
\hrule
\bigskip

\subsection{Soundness}

\renewenvironment{proof}{\noindent{\bf Proof.~}}{}

\begin{theorem}
Let $s=s'$ be an instance of one of the tactics introduced in this section.
let $a_i=b_i$, $i=1..k$, be the proof obligations. If $\semp{a_i}=\semp{b_i}$
for all interpretations of the free variables of $a_i$ and $b_i$, then
$\semp{s}=\semp{s'}$ for all interpretations of the free variables of $s$ and $s'$.
\end{theorem}

\begin{proof}
For the tactics with {\bf Obligations:} tactic, the theorem is trivial.

\medskip
For Stratify, let $f$, $g$ be partial functions such that
\[\renewcommand\arraystretch{1.3}
  \forall \theta,\zeta.\quad \begin{array}{l}f\,(g\,\zeta\,\theta) ~=~ f\,\zeta \\
  g\,(f\,\theta)\,\theta ~=~ h\,\theta
  \end{array}\quad\]
  
Assume that $\zeta = \fix f$ and $\theta = \fix (g\,\zeta)$. That is,
\[\renewcommand\arraystretch{1.3}
  \begin{array}{l}
    f\,\zeta = \zeta\\
    g\,\zeta\,\theta = \theta
  \end{array}\]
  
Then ---
\[\renewcommand\arraystretch{1.3}
  \begin{array}{l@{}l}
   h\,\theta & {}= g\,(f\,\theta)\,\theta = g\,(f\,(g\,\zeta\,\theta))\,\theta = \\
             & {}= g\,(f\,\zeta)\,\theta = \theta
  \end{array}\]
  
So $\theta = \fix h$. We get $\fix h = \fix \big(g \,(\fix f)\big)$, or, equivalently,
\[\fix h = \fix f \applt \psi\mapsto\fix (g\,\psi)\]

Now instantiate $h$, $f$, and $g$, with $f\applt g$, $f$, and $g'$ from \Cref{tactics:Stratify},
and we obtain the equality in the tactic.

\medskip
For Synth, ({\it i}) assume
\[\forall \overline\theta.\quad h\,(f\,\overline\theta)=f\,\overline\theta \quad\]
%
and let $\theta=\theta_{1..r}$ such that $\theta_i=\fix f_i$. So $f_i\,\theta_i=\theta_i$.
Let $\theta=\theta_1::\T_1/\cdots/\theta_r::\T_r$.
\[\renewcommand\arraystretch{1.3}
  \begin{array}{l@{}l}
   f\,\overline\theta & {}= (f_1\,\theta_1)::\T_1 / \cdots / (f_r\,\theta_r)::\T_r =  \\
     & {}= \theta_1::\T_1 / \cdots / \theta_r::\T_r = \theta \\[.5em]
   h\,\theta & {}= h\,(f\,\overline\theta) = f\,\overline\theta = \theta
   \end{array}\qquad\]
   
Then $\theta=\fix h$;\\
We get $\fix h = (\fix f_1)::\T_1 / \cdots / (\fix f_r)::\T_r$,
as required.

({\it ii}) assume
\[\renewcommand\arraystretch{1.3}
  \forall\theta. \quad
  \begin{array}{l}
  \displaystyle
    h\,\theta = \theta ~\limplies \bigwedge_{i=1..r}(f_i\,\theta)::\T_i = \theta::\T_i \\
    \theta = \theta::\T_1 ~\big/~ \cdots ~\big/~ \theta::\T_r
  \end{array}  
\]
%
and let $\theta=\fix h$. So $h\,\theta=\theta$, therefore $(f_i\,\theta)::\T_i=\theta::\T_i$.
Since we assume that $\fix f_i$, we get by induction that $(\fix f_i)::\T_i=\theta::\T_i$.
Hence,
\[(\fix f_1)::\T_1 / \cdots / (\fix f_r)::\T_r = \theta::\T_1 / \cdots / \theta::\T_r = \theta\]
\end{proof}

Our reliance on the termination of $\fix$ expressions may seem conspicuous, since some of these
expressions are generated automatically by the system. However, a closer look reveals that whenever
such a computation is introduced, the set of the recursive calls is makes is a subset of those made by the existing one.
Therefore, if the original recurrence terminates, so does the new one. In any case, all the recurrences
in our development have a trivial termination argument (the indexes $i$,$j$ change monotonically between calls),
so practically, this should never become a problem.
