\section{A Unified Language}

\newcommand\semp[1]{[\![{#1}]\!]}
\newcommand\fix{\operatorname{fix}}

Bellmania uses the same language for specifications and for programs.  Its core is simply-typed
$\lambda$-calculus with universally quantified type variables (polymorphic types).

We write abstraction terms as $(v:\T)\mapsto e$, where $\T$ is the type of the argument $v$ and $e$ is
the body. Curried functions $(v_1:\T_1)\mapsto (v_2:\T_2) \mapsto \cdots \mapsto (v_n:\T_n) \mapsto e$ are abbreviated 
as $(v_1:\T_1)\cdots(v_n:\T_n)\mapsto e$.

The semantics differ slightly from that of traditional functional languages: arrow types $\T_1\to\T_2$
are interpreted as {\bf mappings} from values of type $\T_1$ to values of type $\T_2$. Algebraically,
interpretations of types, $\semp{\T_1}$, $\semp{\T_2}$, are sets, and interpretations of arrow-typed terms,
$f : \T_1\to\T_2$, are {\bf partial functions} --- $\semp{f} : \semp{T_1}\rightharpoonup\semp{T_2}$.
This implies that a term $t : \T$ may evaluate to an \newterm{undefined} value, $\semp{t}=\bot_\T$
(We would shorten it to $\semp{t}=\bot$ when the type is either insignificant or understood from the context).
For simplicity, we shall identify $\bot_{\T_1\to\T_2}$ with the empty mapping $(v:\T_1)\mapsto\bot_{\T_2}$.

\subsection{Liquid Types}

The core is augmented with predicate abstraction in the form of logically quantified data types 
(Liquid Types~\citneeded{}). These are refinement types restricted via a set of abstraction predicates,
called \newterm{qualifiers}, which are defined over the base types.
Some qualifiers are built-in, and more can defined by the user. To keep the syntax simple, we somewhat
limit the use of qualifiers, allowing only the following forms:

\begin{itemize}
  \item $\{v:\T ~|~ P(v)\}$, abbreviated as $\T\cap P$. When the signature of $P$ is known (which is
  almost always), it is enough to write $P$.
  \item $\{v:\T ~|~ P(v)\land Q(v)\}$, abbreviated as $\T\cap P\cap Q$, or just $P\cap Q$. This extends
  to any number of conjuncts of the same form.
  \item $x : \T_2 \to \{v:\T_2 ~|~ R(x,v)\} \to \T_3$, abbreviated as $\big((\T_1\times\T_2)\cap R\big)\to\T_3$.
  Note that the first parameter to $R$ must be the preceding argument. This extends to quantifiers of
  any arity. Also note that the language does not define tuple types; hence there is no distinction
  between curried and uncurried function types.
\end{itemize}

The type refinement operators $\cap$ and $\times$ may be composed to create conjunctions of qualifiers,
as long as their argument sets are either disjoint or contained, but not overlapping;
for example, \[x:\{v:\T_1~|~P(v)\}\to \{v:\T_2~|~Q(v)\land R(x,v)\}\to\T_3\] can be written as
$\big((P\times Q)\cap R\big)\to\T_3$, but \[x:T_1\to y:\{v:\T_2~|~R(x,v)\} \to \{v:\T_3~|~R(y,v)\}\to\T_4\]
cannot be represented by this restricted fragment.

As with any refinement type system, we define the \newterm{shape} of a type $\T$ to be the raw type
obtained from it by removing all qualifiers.

\subsection{Operators}

\newcommand\applt{\textrm{{\scriptsize\,}\guillemotright{\scriptsize\,}}}

{\color{Gray}
\begin{itemize}
  \item Fixed point operator $\fix f$
  \item Slash operator $/$
  \item Cast operator $t :: \T$ inside expressions (and $t|_{\setlength{\fboxsep}{2pt}\fbox{}}$)
  \item Left application $x \applt f$
\end{itemize}
}

\subsection{Typing Rules}

We define a Bellmania program to be well-typed iff its \newterm{raw form}, obtained by replacing all types by
their shapes, is well-typed. This gives a simple characteristic of type safety without the need to
explictly write any new typing rules. It also means that for $f:\T_1\to\T_2$, $x:T_3$, then $f\,x:\T_2$ whenever
$\T_1$ and $\T_3$ have the same shape. This is much more permissive than the original Liquid Types,
and requires some explanation.\todo{covariance and coercion}

\subsection{Type Inference}

\subsection{Primitives}

The standard library contains some common primitives:

\begin{itemize}
  \item ${+}, {-} : \forall \T.~\T\to\T\to\T$ --- polymorphic binary operators.
  \item ${<} : \forall \T. ~\T\to\T\to\B$ --- a polymorphic order relation.
  \item $\min, \max, \Sigma : \forall \T\,\S.~ (\T\to\S)\to\S$ --- reduction operators
    on ordered/unordered sequences\todo{bags? multisets?}. The sequence is represented by a mapping $f : \T\to\S$,
    so that e.g. \[\semp{\min f} = \min \{\semp{f}\,v \;|\; v\in\semp{A}, \semp{f}\,v\neq\bot\}\]
    The sequences are expected to be finite.
\end{itemize}

\subsection{Example \hrulefill}
\hrule
\bigskip
