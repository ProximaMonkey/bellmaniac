\documentclass[preprint]{sigplanconf}

\begin{document}


\special{papersize=8.5in,11in}
\setlength{\pdfpageheight}{\paperheight}
\setlength{\pdfpagewidth}{\paperwidth}

%\conferenceinfo{CONF 'yy}{Month d--d, 20yy, City, ST, Country} 
%\copyrightyear{20yy} 
%\copyrightdata{978-1-nnnn-nnnn-n/yy/mm} 
%\doi{nnnnnnn.nnnnnnn}

%\titlebanner{banner above paper title}        % These are ignored unless
%\preprintfooter{short description of paper}   % 'preprint' option specified.

\title{Bellmania: Deriving Implementations from Specifications by Transformation}

%\authorinfo{Shachar Itzhaky}
%           {MIT CSAIL}
%           {shachari@mit.edu}
\authorinfo{Double-blind submission}
           {}
           {}

\maketitle

\begin{abstract}
Programmers almost never write optimal code on the first phase.
Instead, they usually begin with a simplified, possibly inefficient, or even
na\"ive implementation. When the result is satisfactory, they would move
on to revise the code in order to improve performance. These normally
change the code considerably compared to the original version. It is especially
true for code that targets multi-core processors, since many considerations
must be taken into account for parallel code in order to utilize all available
cores.

Clearly, each iteration becomes more tedious as the code increases in complexity,
and naturally holds more chance of introducing defects. The programmer must
exercise care to make sure that each version is functionally equivalent to the
previous one. This is when mechanical reasoning can prove useful, by 
(a) packaging familiar techniques as routines, thus avoiding repetitive work, and
(b) checking each modification step, warning the developer when functionality
may be accidentally changed.

We present Bellmania, a framework comprising of a language for specifying
dynamic programming algorithms as recurrences,
and a calculus that facilitates gradual transformation of these 
specifications into efficient implementations.
In particular, it supports using the Divide and Conquer technique to derive
parallel implementations that optimally utilize memory caches.
\end{abstract}

\category{CR-number}{subcategory}{third-level}

\keywords
synthesis, dynamic programming, smt

\end{document}
